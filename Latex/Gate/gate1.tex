\documentclass[10pt,-letter paper]{article}
\usepackage[left=1in, right=0.75in, top=1in, bottom=0.75in]{geometry}
\usepackage{graphicx} % Required for inserting images
\usepackage{siunitx}
\usepackage{setspace}
\usepackage{gensymb}
\usepackage{xcolor}
\usepackage{caption}
%\usepackage{subcaption}
\doublespacing
\singlespacing
\usepackage[none]{hyphenat}
\usepackage{amssymb}
\usepackage{relsize}
\usepackage[cmex10]{amsmath}
\usepackage{mathtools}
\usepackage{amsmath}
\usepackage{commath}
\usepackage{amsthm}
\interdisplaylinepenalty=2500
%\savesymbol{iint}
\usepackage{txfonts}
%\restoresymbol{TXF}{iint}
\usepackage{wasysym}
\usepackage{amsthm}
\usepackage{mathrsfs}
\usepackage{txfonts}
\let\vec\mathbf{}
\usepackage{stfloats}
\usepackage{float}
\usepackage{cite}
\usepackage{cases}
\usepackage{subfig}
%\usepackage{xtab}
\usepackage{longtable}
\usepackage{multirow}
%\usepackage{algorithm}
\usepackage{amssymb}
%\usepackage{algpseudocode}
\usepackage{enumitem}
\usepackage{mathtools}
%\usepackage{eenrc}
%\usepackage[framemethod=tikz]{mdframed}
\usepackage{listings}
%\usepackage{listings}
\usepackage[latin1]{inputenc}
%%\usepackage{color}{   
%%\usepackage{lscape}
\usepackage{textcomp}
\usepackage{titling}
\usepackage{hyperref}
%\usepackage{fulbigskip}   
\usepackage{tikz}
\usepackage{graphicx}
\lstset{
  frame=single,
  breaklines=true
}
\let\vec\mathbf{}
\usepackage{enumitem}
\usepackage{graphicx}
\usepackage{siunitx}
\let\vec\mathbf{}
\usepackage{enumitem}
\usepackage{graphicx}
\usepackage{enumitem}
\usepackage{tfrupee}
\usepackage{amsmath}
\usepackage{amssymb}
\usepackage{mwe} % for blindtext and example-image-a in example
\usepackage{wrapfig}
\graphicspath{{figs/}}
\providecommand{\cbrak}[1]{\ensuremath{\left\{#1\right\}}}
\providecommand{\brak}[1]{\ensuremath{\left(#1\right)}}
\newcommand{\sgn}{\mathop{\mathrm{sgn}}}
\providecommand{\abs}[1]{\left\vert#1\right\vert}
\providecommand{\res}[1]{\Res\displaylimits_{#1}} 
\providecommand{\norm}[1]{\left\lVert#1\right\rVert}
%\providecommand{\norm}[1]{\lVert#1\rVert}
\providecommand{\mtx}[1]{\mathbf{#1}}
\providecommand{\mean}[1]{E\left[ #1 \right]}
\providecommand{\fourier}{\overset{\mathcal{F}}{ \rightleftharpoons}}
%\providecommand{\hilbert}{\overset{\mathcal{H}}{ \rightleftharpoons}}
\providecommand{\system}{\overset{\mathcal{H}}{ \longleftrightarrow}}
	%\newcommand{\solution}[2]{\textbf{Solution:}{#1}}
%\newcommand{\solution}{\noindent \textbf{Solution: }}
\newcommand{\cosec}{\,\text{cosec}\,}
\providecommand{\dec}[2]{\ensuremath{\overset{#1}{\underset{#2}{\gtrless}}}}
\newcommand{\myvec}[1]{\ensuremath{\begin{pmatrix}#1\end{pmatrix}}}
\newcommand{\myaugvec}[2]{\ensuremath{\begin{amatrix}{#1}#2\end{amatrix}}}
\newcommand{\mydet}[1]{\ensuremath{\begin{vmatrix}#1\end{vmatrix}}}
\title{}
\usepackage{circuitikz}
\title{Gate Question}
\author{SECTION A}
\date{\today}
\begin{document}

\maketitle

\begin{enumerate}

	\item In the circuit shown, the clock frequency, i.e., the frequency of the clk signal, is 12 KHz. The frequency of the signal at $\mathbf{Q2}$ is $\underline{\hspace{18pt}}$ KHz.
		\hfill(GATE-EC2019,25)
\end{enumerate}
		\begin{figure}[H]
			\begin{circuitikz}

\draw (7,2)coordinate (E) -- (9,2)coordinate (F) -- (9,-1)coordinate (G) -- (7,-1)coordinate (H) -- (7,2)coordinate (E);
\draw (11,2)coordinate (I) -- (13,2)coordinate (J) -- (13,-1)coordinate (K) -- (11,-1)coordinate (L) -- (11,2)coordinate (I);
% AND Gate
\draw (6,1.5) node[and port] (and) {};
\draw (and.in 1) node[left] {};
\draw (and.in 2) node[left] {};
\draw (and.out) |- ($(E)!0.2!(H)$)--++(0:0)node[right]{$D1$}; % and gate output is connected to d1.

%clock
\draw ($(L)!0.5!(K)$)node[anchor=south]{$clk$};
\draw ($(H)!0.5!(G)$)node[anchor=south]{$clk$};
\draw(6,-2) node[above]{$12 KHz$} |- (12,-2);
\draw ($(H)!0.5!(G)$)node[anchor=south,xshift=6]{}--++(90:-1)--++(0:0)node[left]{};
\draw ($(L)!0.5!(K)$)node[anchor=south,xshift=6]{}--++(90:-1)--++(0:0)node[left]{};

\draw($(F)!0.2!(G)$)node[left]{$Q1$} -- ($(I)!0.2!(L)$)node[right]{$D2$}; % q1 is connected to d2.
    
\draw($(F)!0.8!(G)$)node[left]{$\overline{Q1}$};
    
\draw($(J)!0.2!(K)$)node[left]{$Q2$};
    
\draw($(J)!0.8!(K)$)node[left]{$\overline{Q2}$};
    
    
\draw ($(J)!0.2!(K)$)--++(0:1)node[right]{};
    
\draw(and.in 1) --++(90:2)-|(9.5,1)|-($(F)!0.8!(G)$)(0:0)node[left]{}; % and gate input 1 is connected to q1 bar.
    
\draw(and.in 2) --++(-90:4)-|(13.5,-0.5)|- ($(J)!0.8!(K)$);  % and gate input 2 is connected to q2 bar.
    
\end{circuitikz}

			\caption{Circuit Daigram}
		\end{figure}
\end{document}

